\hypertarget{section-technical-risks}{%
\section{Risks and Technical Debts}\label{section-technical-risks}}

\subsection{Death by a 1000 microservices}
As a development team, we are fully aware of the significant costs associated with transitioning to a microservices architecture. In doing so, we acknowledge that we have opened Pandora’s box for what was initially a relatively straightforward problem. This decision brings with it increased technical debt, heightened development complexity, and greater challenges in maintenance and operational overhead.

Despite these risks, we have consciously chosen this path, as one of the primary goals of our project is to build a real-world system that allows us to confront and learn from the challenges faced in real-world scenarios. We understand that the scalability requirements of our project will likely never approach the levels that would justify adopting microservices. However, this project is not just about achieving scalability; it’s about gaining hands-on experience with the hurdles and intricacies of designing, developing, and operating a distributed system.

\subsection{Code Quality}
Given the complexity of the system we are introducing, we are fully aware that maintaining a high standard of code quality will be a significant challenge. One of the immediate trade-offs we recognize is that the added architectural overhead will limit our ability to write comprehensive tests. Without sufficient test coverage, the risk of regressions and unforeseen issues increases, especially as the system evolves.

This challenge is further compounded by the varying skill levels across the team. We acknowledge that this disparity will likely result in a codebase that feels fragmented—like it was written by multiple developers with differing styles, approaches, and levels of experience. While this is a natural consequence of a collaborative learning environment, it also adds another layer of complexity to maintaining consistency and readability in the code.

Despite these challenges, we see this as an opportunity for growth. By addressing these issues directly—through code reviews, mentorship, and shared learning—we aim to bridge the skill gaps and gradually bring more cohesion to the project. We also understand the importance of balancing our desire to explore real-world challenges with the need to build a system that remains functional, maintainable, and as unified as possible under the circumstances.

\subsection{Too many hamers for a single nail}
We recognize that our approach could be viewed as using too many hammers for a single nail, and this concern has already been highlighted in our earlier discussions. As a development team, we are fully aware of the technical debt we are incurring by relying on a multitude of external systems. This dependency not only adds complexity but also introduces long-term risks. Software systems, by their very nature, either stagnate and become obsolete or require breaking updates that ripple across the architecture, potentially causing instability.

Moreover, in today’s environment, where security vulnerabilities are discovered and exploited at an alarming pace, implementing daily or frequent patches becomes a non-negotiable requirement. In such a context, this design could become a significant operational burden, particularly if we ever aimed to release this product into the wild. Managing and securing a system with so many moving parts would demand substantial ongoing resources, from continuous monitoring to rapid patch deployment and testing across all components.
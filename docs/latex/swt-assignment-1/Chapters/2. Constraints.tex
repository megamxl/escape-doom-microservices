\hypertarget{section-architecture-constraints}{%
\section{Architecture
Constraints}\label{section-architecture-constraints}}

\subsection{Technical Constraints}
\label{subsection:technical-constraints}

\begin{table}[htbp]
    \centering
    \begin{tabularx}{1\textwidth} {
        | >{\raggedright\arraybackslash}X
        | >{\raggedleft\arraybackslash}X | }
        \hline
        Constraint & Background and/or motivation \\
        \hline
        Microservice Architecture & The system must follow a microservice architecture to enable modularity, scalability, and independent deployment of services. \\
        \hline
        Frontend Framework & The frontend must use React/Next.js for server-side rendering and better SEO support. \\
        \hline
        Implementation in Java & The backend uses Java Spring Boot MVC due to its ecosystem support and developer familiarity. The code is written on based on Java 17 \\
        \hline
        Message Queuing & Use of a message broker like RabbitMQ/Kafka for asynchronous communication \\
        \hline
        API Communication & RESTful APIs must be the primary communication method between the frontend and backend \\
        \hline
        Third-party software freely available & If third-party software is involved (for example, a graphical front end), this should ideally be freely available and free of charge. This way the threshold to use it is kept low \\
        \hline
         Deployable to Kubernetes & Since the goal is to build a microservice architecture Kubernetes is the orchestrator chosen to handle the complex deployment of the system. \\
         \hline
    \end{tabularx}
    \caption{Technical Constraints}
    \label{tab:technical-constraints}
\end{table}

\subsection{Organizational Constraints}
\label{subsection:organizational-constraints}
\begin{table}[h!tbp]
    \centering
    \begin{tabularx}{1\textwidth} {
        | >{\raggedright\arraybackslash}X
        | >{\raggedleft\arraybackslash}X | }
        \hline
         Team & Kambal, Nowak, Perov, Selbach, Winter are tackling the problem with the guidance of Doz. DI DI Mag. Dr. Karl M. Göschka \\
        \hline
         Time Table & The planning phase began in November 2024, with the rearchitecture plan scheduled to be completed by the end of January 2025. Following the planning phase, the implementation will take place over the course of the second semester of 2025. The objective is to have a fully functioning system ready before the semester ends, allowing for an acceptance test to be conducted during a lecture. \\
         \hline
         Development approach & The team works risk driven and incremental. The main goal of this approach would be to have an CI/CD deployment cycle in place that each change is deliverable without overhead. \\
         \hline
         Development Tools & Version control is handled through git and published to Github. The provided code is tested through the GitHub action capabilities. For IDE'S the jet brains suit is recommended but not mandatory. For development docker is used to mange all systems EscapeDoom is depending on. \\
         \hline 
         Open Source license & All source code is published under Apache 2.0 license to enable every university to use this System for their own curriculum. \\
         \hline
    \end{tabularx}
    \caption{Organizational Constraints}
    \label{tab:organizational-constraints}
\end{table}

\newpage

\subsection{Conventions}
\label{subsection:conventions}

\begin{table}[h!tbp]
    \centering
    \begin{tabularx}{1\textwidth} {
        | >{\raggedright\arraybackslash}X
        | >{\raggedleft\arraybackslash}X | }
        \hline
        Convention & Background and/or motivation \\
        \hline
        Architecture documentation & Terminology and structure according to the arc42 template. \\
        \hline
        Coding guidelines for Java & The system must be refactored according to \href{https://vaadin.com/docs/latest/building-apps/project-structure/multi-module}{Java and Maven multi-module standards}. \\
        \hline
        Coding guidelines for Next.js / React & The system must be refactored according to \href{https://github.com/dwarvesf/nextjs-boilerplate/blob/master/docs/CODE_STYLE.md}{Next.js standards}. \\
        \hline
        Coding guidelines for Go & The system must be refactored according to \href{https://github.com/golang-standards/project-layout}{Go project layout standard.}. \\
        \hline
    \end{tabularx}
    \caption{Conventions}
    \label{tab:conventions}
\end{table}